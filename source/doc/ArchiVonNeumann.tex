\documentclass{beamer}
 
\usepackage[utf8]{inputenc}
\usetheme{Madrid}
%\usetheme{Bergen}
\usecolortheme{beaver}
 
%Information to be included in the title page:
\title{Architecture de Von Neumann}
\author{Yannick CHISTEL}
\institute{Lycée Dumont d'Urville - CAEN}
\date{Septembre 2019}
 

 
\begin{document}
 
\frame{\titlepage}
 
\begin{frame}
\frametitle{Architecture d’un ordinateur}
Un ordinateur est composé de plusieurs unités fonctionnelles:
\begin{itemize}
\item La mémoire centrale
\item L’unité centrale ou processeur (CPU)
\item Les unités d’E/S
\end{itemize}
Cet ensemble d’unités constituent le système informatique

\end{frame}
 
\begin{frame}
\frametitle{Mémoire centrale d’un ordinateur}
\begin{block}{Définition}
La mémoire centrale (RAM) contient les programmes à exécuter et les données du programme. 

Elle peut se représenter comme un tableau de cases mémoires appelées \textbf{mots} mémoires.
\end{block}

\begin{itemize}
\item La taille des mots mémoire peut varier de 8 à 64 \textit{bits}.
\item Chaque case ou mot mémoire est repéré par une \textbf{adresse} unique.
\item L'accès au contenu du mot mémoire est en \textbf{lecture} ou \textbf{écriture}.
\item La mémoire centrale est volatile. Lorsqu'il n'y a plus de tension, elle est effacée.
\end{itemize}
Il existe des mémoires persistantes dites mémoires de masse.
\end{frame}

\begin{frame}
\frametitle{Le processeur (CPU)}

\begin{block}{Définition}
Un processeur se compose en 2 parties:
\begin{itemize}
\item L'unité arithmétique et logique : effectue les opérations mathématiques et les opérations logiques.
\item L’unité de contrôle : joue le rôle d'un chef d'orchestre. Il charge les instructions et les données contenues en mémoire et les envoie à l'UAL pour le traitement.
\end{itemize}

\end{block}

Un processeur contient ses propres mémoires appelées \textbf{registres}. 
\begin{itemize}
\item Certains registres contiennent les données et les instructions du programme en binaire,
\item D'autres registres contiennent les adresses mémoires des données et des instructions du programme.
\end{itemize}
Il existe une mémoire cache qui permet de limiter l'accès à la mémoire centrale.
\end{frame}

\begin{frame}
\frametitle{Les unités d'entrée-sortie E/S}

Il existe de nombreux périphériques d'entrée et sortie :

\begin{enumerate}
\item Les périphériques d'entrée :

\begin{itemize}
\item Les périphériques de saisies comme le clavier et la souris;
\item Les manettes de jeu, les lecteurs de code (code barre, qrcode, etc.)
\item Les scanners, les appreils photos, les webcams, etc.
\end{itemize}

\item Les périphériques de sortie :
\begin{itemize}
\item les écrans et vidéo-projecteurs,
\item les imprimantes,
\item les hauts parleurs, etc.
\end{itemize}

\item Les périphériques d'entrée et sortie:
\begin{itemize}
\item les lecteurs de disques (CD, Blue Ray, etc.)
\item les disques durs, les clés USB ou les cartes SD,
\item les cartes réseaux, etc.
\end{itemize}
\end{enumerate}
\end{frame}


\begin{frame}
\frametitle{Liaisons entre les unités}
\begin{block}{Définition}
Les différentes unités fonctionnelles sont reliées entre elles, soit par des circuits intégrés comme la carte mère, soit par des liaisons filaires électriques.

Ces liaisons sont appelées des \textbf{bus}. Il existe trois type de bus:
\begin{itemize}
\item Les bus de données qui transportent les données et instructions des programmes,
\item Les bus d'adressage qui transportent les adresses mémoires utilisées,
\item Les bus de contrôle pour indiquer s'il s'agit d'une lecture ou d'une écriture en mémoire.
\end{itemize}
\end{block}

\begin{block}{Remarque}
Le processeur contient aussi une \textbf{horloge} qui rythme l'exécution des programme et les échanges entre le processeur et la mémoire.
\end{block}
\end{frame}

\begin{frame}
\frametitle{Schéma de l'architecture de Von Neumann}
Les ordinateurs suivent cette même architecture créée par le professeur Von Neumann.

\begin{center}
\includegraphics[scale=0.5]{archi-von-neumann.png}
\end{center}
\end{frame}
\end{document}

