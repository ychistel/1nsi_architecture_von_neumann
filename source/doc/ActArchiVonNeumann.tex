\documentclass[11pt,a4paper]{article}

\usepackage{style2017}
\usepackage{hyperref}

\hypersetup{
    colorlinks =false,
    linkcolor=blue,
   linkbordercolor = 1 0 0
}
\newcounter{numexo}
\setcellgapes{1pt}

\begin{document}


\begin{NSI}
{Activité}{Architecture Von Neumann - Langage machine}
\end{NSI}


\section{Architecture von Neumann}

Les appareils numériques comme les ordinateurs, tablettes ou smartphones sont construits selon un modèle d'architecture créé par Von Neumann. Celle-ci se décompose en trois parties:
\begin{itemize}
\item La mémoire centrale
\item Le microprocesseur
\item Les bus de communication pour transporter les informations.
\end{itemize}

Vous allez étudier cette architecture en vous basant sur 2 articles du site \textbf{https://interstices.info}
\begin{enumerate}
\item Le modèle d'architecture de Von Neumann
\item Mémoire et unité centrale, un couple dédié à l'exécution des programmes
\end{enumerate}


\subsection*{Schéma de l'architecture}

Dessiner le schéma du modèle originel de Von Neumann pour l'architecture des ordinateurs. \vspace{8cm}

\subsection*{La mémoire centrale}

\begin{enumerate}
\item Quel est le type d'information stocké dans la mémoire ? \vspace{2cm}
\item Quel est la capacité en octet/bits de cette cellule ?\vspace{2cm}
\item Comment est identifiée la cellule ? \vspace{2cm}

\item Quelle est la persistance de la mémoire ?  Comment est-elle désignée ?\vspace{2cm}

\item Comment se caractérise la mémoire ? \vspace{3cm}
\end{enumerate}


\subsection*{Le processeur}

\begin{enumerate}
\item Comment peut-on qualifier le rôle du processeur ?\vspace{2cm}
\item Quels sont les 4 éléments principaux composant un processeur ? Quels sont leurs rôles ?\vspace{10cm}
\end{enumerate}





\newpage
\section{Langage machine}


\begin{enumerate}
\item Qu'est-ce que le langage machine ? \vspace{3cm}
\item Qu'est-ce que le langage assembleur ? \vspace{3cm}
\item Comment un programme écrit en langage de haut niveau est-il transformer en langage machine ?\vspace{3cm}

\item Il existe sur le web un simulateur pour réaliser des instructions en langage machine à l'adresse : \\
\textsf{https://www.peterhigginson.co.uk/AQA/}\medskip

Vous trouverez des informations sur l'utilisation de cet émulateur sur le site \\
\textsf{https://pixees.fr/informatiquelycee/n\_site/nsi\_prem\_sim\_cpu.html}\medskip

Indiquer ce que font les différentes instructions écrites en assembleur.\medskip
\begin{enumerate}

\item ADD R0, R1, \#25 \vspace{1.5cm}
\item LDR R2,64 \vspace{1.5cm}
\item MOV R3, \#45 \vspace{1.5cm}
\item STR R4, 72 \vspace{1.5cm}
\item SUB R5,R2,R3 \vspace{1.5cm}
\item CMP R3, \#25 \\
\hspace{0.5cm}BGT 15 \vspace{1.5cm}
\end{enumerate}

\item Écrire les instructions en assembleur correspondant aux phrases suivantes:
\begin{itemize}
\item Place la valeur 15 dans le registre R0 \vspace{1cm}
\item Place la valeur 7 dans le registre R1 \vspace{1cm}
\item Additionne la valeur stockée dans le registre R0 et la valeur stockée dans le registre R1, le résultat est stocké dans le registre R5 \vspace{1cm}
\item Place le contenu du registre R5 à l'adresse mémoire 125. \vspace{1cm}
\item Place la valeur 10 dans le registre R1 \vspace{1cm}
\item Place la valeur stockée à l'adresse mémoire 125 dans le registre R0 \vspace{1cm}
\item Soustrait la valeur stockée dans le registre R0 et la valeur stockée dans le registre R1, le résultat est stocké dans le registre R5 \vspace{1cm}
\item Place le contenu du registre R5 à l'adresse mémoire 125. \vspace{1cm}
\end{itemize}
\item Quelles sont les valeurs dans les différents registres à l'issu de ce programme ? \vspace{2cm}
\item Saisir votre programme dans le simulateur et vérifier vos réponses.
\end{enumerate}



\end{document}

